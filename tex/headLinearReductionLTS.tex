\documentclass[a4paper, 10pt]{article}
\usepackage[margin=0.5in]{geometry}
\usepackage[utf8]{inputenc}%кодировка
\usepackage[russian]{babel}%используем русский и английский языки с переносами
\usepackage{cite}
\usepackage{multirow}
\usepackage{float}
%\usepackage{tikz}
%\usetikzlibrary{arrows}
\usepackage{amssymb }
\usepackage{amsmath}
\usepackage{listings}
\usepackage{cancel}
%\usepackage{xcolor}
%\graphicspath{{pictures/}}
%\definecolor{dkgreen}{rgb}{0,0.6,0}
%\definecolor{dred}{rgb}{0.545,0,0}
%\definecolor{dblue}{rgb}{0,0,0.545}
%\definecolor{lgrey}{rgb}{0.9,0.9,0.9}
%\definecolor{gray}{rgb}{0.4,0.4,0.4}
%\definecolor{darkblue}{rgb}{0.0,0.0,0.6}

%\usepackage{pythontex}
%\usepackage{minted}

% for printing trees
\usepackage{verbatim}

%\usepackage[labelformat=empty]{caption}
%\usepackage{fontspec}
%\usepackage{polyglossia}
%\setdefaultlanguage{english}
%\defaultfontfeatures{Ligatures=TeX}
%\setmainfont{CMU Serif}
%\setsansfont{CMU Sans Serif}
%\setmonofont{CMU Typewriter Text}

%\usepackage{stmaryrd}
\usepackage{amsfonts}
\newcommand\abs[1]{\left|#1\right|}

%\usepackage{tikz}
\newcommand{\tikzmark}[3][]{\tikz[remember picture,baseline] \node [inner xsep=0pt,anchor=base,#1](#2) {#3};}
%\usepackage{lscape}
%\usepackage{pdflscape}

\lstset{
  frame=none,
  xleftmargin=2pt,
  stepnumber=1,
  numbers=left,
  numbersep=5pt,
  numberstyle=\ttfamily\tiny\color[gray]{0.3},
  belowcaptionskip=\bigskipamount,
  captionpos=b,
  escapeinside={*'}{'*},
  language=haskell,
  tabsize=2,
  emphstyle={\bf},
  commentstyle=\it,
  stringstyle=\mdseries\rmfamily,
  showspaces=false,
  keywordstyle=\bfseries\rmfamily,
  columns=flexible,
  basicstyle=\small\sffamily,
  showstringspaces=false,
  morecomment=[l]\%,
}

\newcommand{\State}[1]{\left<{#1}\right>}
\newcommand{\InContext}[2]{{#1}\left[{#2}\right]}
\newcommand{\RuleNo}[1]{\eqno[\textsc{#1}]}
\newcommand{\Rule}[2]{{#1}\longrightarrow{#2}}

%\usetikzlibrary{calc}

\begin{document}

\section{Labelled Transition System for repeated Head Linear Reduction}

\subsection{Notes}
State is a tuple $\langle$ $\lambda$-term with underlined node, context, list of arguments $\rangle$, where
\begin{itemize}
\item $\lambda$-term (a tree; by considering $\lambda$-term as a tree it becomes possible to cross arguments out of tree with a parent application node (denotes as $A_{\xcancel{B}}$)); underlined node is a usual lambda term with one underlined position;
\item Context $\Gamma$ is an unordered list of pairs ($variable : term$);
\item List of arguments $\Delta$ is an ordered list of $\lambda$-terms (one can also think about $\Delta$ as an ordered list of pointers to the corresponding subtree);
\item Term that is received by throwing out all elements that are crossed out is a \textit{residual term in the normal form};
\item \textit{Initial} configuration is $\langle \lambda-term\ with\ underlined\ root,\ \emptyset,\ \$ \rangle$;
\item \textit{Final} configuration is $\langle \InContext{M}{\underline{x}},\ \Gamma,\ \$ \rangle$.
\end{itemize}


\subsection{Rules}

$$
\Rule{\State{\InContext{A}{e_1\underline{@}e_2};\;\Gamma;\;\Delta}}
     {\State{\InContext{A}{\underline{e_1}@e_2};\;\Gamma;\;e_2\bullet\Delta}}
\RuleNo{App}
$$

$$
\Rule{\State{\InContext{A}{\underline{\lambda x}.e};\;\Gamma;\;\$\bullet\Delta}}
     {\State{\InContext{A}{\lambda x.\underline{e}};\;\Gamma;\;\$\bullet\Delta}}
\RuleNo{Lam-Non-Elim}
$$

$$
\Rule{\State{\InContext{A}{\underline{\lambda x}.e};\;\Gamma;\;B\bullet\Delta}}
     {\State{\InContext{A_{\xcancel{B}}}{\xcancel{\lambda x}.\underline{e}};\;x:B,\,\Gamma;\;\Delta}},\;B\ne\$
\RuleNo{Lam-Elim}
$$

$$
\Rule{\State{\InContext{A}{\underline{x}};\;x:B,\,\Gamma;\;\Delta}}
     {\State{\InContext{A}{\underline{B}};\;x:B,\,\Gamma;\;\Delta}}
\RuleNo{BVar}
$$

$$
\Rule{\State{\InContext{A}{\InContext{M}{\underline{x}}@B};\;\Gamma;\;B\bullet\$\bullet\Delta}}
     {\State{\InContext{A}{M[x]@\underline{B}};\;\Gamma;\;\$\bullet\Delta}},\;B\ne\$,\;x\notin dom\,\Gamma
\RuleNo{FVar-0}
$$

$$
\Rule{\State{\InContext{A}{\InContext{M}{\underline{x}}@B};\;\Gamma;\;B\bullet C\bullet\Delta}}
     {\State{\InContext{A}{M[x]@\underline{B}};\;\Gamma;\;\$\bullet C\bullet\Delta}},\;C\ne\$,\;B\ne\$,\;x\notin dom\,\Gamma
\RuleNo{FVar-1}
$$

$$
\Rule{\State{\InContext{A}{\InContext{M}{\underline{x}}@B};\;\Gamma;\;\$\bullet B\bullet\Delta}}
     {\State{\InContext{A}{\InContext{M}{x}@\underline{B}};\;\Gamma;\;\Delta}},\;B\ne\$,\;x\notin dom\,\Gamma
\RuleNo{FVar-2}
$$


\subsection{Correctness proof}

\subsubsection{Residual term $\InContext{M}{\underline{x}}$ has no redexes}
\paragraph{Proof}
We will prove by induction (with step size equal to 2) on number of steps that there is no way for redexes to appear in a path from the root to the current underlined node:
\begin{itemize}
\item \textbf{Base} $\;$ Straightforward:
  \begin{enumerate}
  \item $n == 0$, empty term has no redexes;
  \item $n == 1$, path from root to itself contains no redexes.
  \end{enumerate}
\item \textbf{Induction Hypothesis} There is no redexes for all paths with a length less than $n$ ($2 \leq n$).
\item \textbf{Induction Step ()}\\
  Redex is an \textit{application node with $\lambda$-expression as a left child}. We will show that there is no way to construct such subtree without crossing application node out by case analisys on a current (the underlined one) node form:
  \begin{itemize}
  \item For $[Lam-*],\ [BVar],\ [FVar-*]$ rules this is obvious: they do not add $\lambda$ as a left or down child of some node.
  \item The only interesting case is $[App]$ rule. I.e. the rule that was applied on previous step is an $[App]$-rule.
    Here we have two possibilities:
    \begin{itemize}
    \item $e_1 \ne \lambda\ x$; in this case no redexes appear.
    \item $e_1 == \lambda\ x$, for some $x$. In this case a redex appears but by $[App]$-rule we have:\\
      $\Rule{\State{\InContext{A}{e_1\underline{@}e_2};\;\Gamma;\;\Delta}}
       {\State{\InContext{A}{\underline{e_1}@e_2};\;\Gamma;\;e_2\bullet\Delta}}$.\\
      Note, that there \textit{exists} a rule that can be applied next --- $[Lam-Elim]$, and no other rule can be applied. This rule will cross application node with its right subterm out.\\
    \end{itemize}
  \end{itemize}
  Thus, no new redexes can appear.
\end{itemize}

\subsubsection{Only $\langle \InContext{M}{\underline{x}},\ \Gamma,\ \$ \rangle$ can be a final state}
\paragraph{Proof}
\begin{itemize}
\item It is easy to see that the state $\langle \InContext{M}{\underline{x}},\ \Gamma,\ \$ \rangle$ is a final state: there is no rules that can be applied.
\item Let's prove that there is no other final states (with respect to the form of the state). Consider another state which either :
  \begin{itemize}
  \item If current term is not equal to $\InContext{M}{\underline{x}}$ then either $[App]$ or $[Lam-*]$ can be applied.
  (together they have no restrictions on $\Gamma$ and $\Delta$)
  \item If $\Delta \ne \$$. Note that $\Delta$ can not be empty because there is no rule that can move $\$$ that is on the bottom of it. Thus, $\Delta = A \dots \$$ where either $A == \$$ or $A \ne \$$. In this case one of $[FVar-*]$ rules can be applied.
  \end{itemize}
\end{itemize}











\end{document}