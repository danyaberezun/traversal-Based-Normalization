\documentclass[a4paper, 10pt]{article}
\usepackage[margin=0.5in]{geometry}
\usepackage[utf8]{inputenc}%кодировка
\usepackage[russian]{babel}%используем русский и английский языки с переносами
\usepackage{cite}
\usepackage{multirow}
\usepackage{float}
%\usepackage{tikz}
%\usetikzlibrary{arrows}
\usepackage{amssymb }
\usepackage{amsmath}
\usepackage{listings}
\usepackage{cancel}
%\usepackage{xcolor}
%\graphicspath{{pictures/}}
%\definecolor{dkgreen}{rgb}{0,0.6,0}
%\definecolor{dred}{rgb}{0.545,0,0}
%\definecolor{dblue}{rgb}{0,0,0.545}
%\definecolor{lgrey}{rgb}{0.9,0.9,0.9}
%\definecolor{gray}{rgb}{0.4,0.4,0.4}
%\definecolor{darkblue}{rgb}{0.0,0.0,0.6}

%\usepackage{pythontex}
%\usepackage{minted}

% for printing trees
\usepackage{verbatim}

%\usepackage[labelformat=empty]{caption}
%\usepackage{fontspec}
%\usepackage{polyglossia}
%\setdefaultlanguage{english}
%\defaultfontfeatures{Ligatures=TeX}
%\setmainfont{CMU Serif}
%\setsansfont{CMU Sans Serif}
%\setmonofont{CMU Typewriter Text}

%\usepackage{stmaryrd}
\usepackage{amsfonts}
\newcommand\abs[1]{\left|#1\right|}

%\usepackage{tikz}
\newcommand{\tikzmark}[3][]{\tikz[remember picture,baseline] \node [inner xsep=0pt,anchor=base,#1](#2) {#3};}
%\usepackage{lscape}
%\usepackage{pdflscape}

\lstset{
  frame=none,
  xleftmargin=2pt,
  stepnumber=1,
  numbers=left,
  numbersep=5pt,
  numberstyle=\ttfamily\tiny\color[gray]{0.3},
  belowcaptionskip=\bigskipamount,
  captionpos=b,
  escapeinside={*'}{'*},
  language=haskell,
  tabsize=2,
  emphstyle={\bf},
  commentstyle=\it,
  stringstyle=\mdseries\rmfamily,
  showspaces=false,
  keywordstyle=\bfseries\rmfamily,
  columns=flexible,
  basicstyle=\small\sffamily,
  showstringspaces=false,
  morecomment=[l]\%,
}

\newcommand{\State}[1]{\left<{#1}\right>}
\newcommand{\InContext}[2]{{#1}\left[{#2}\right]}
\newcommand{\RuleNo}[1]{\eqno[\textsc{#1}]}
\newcommand{\Rule}[2]{{#1}\longrightarrow{#2}}

%\usetikzlibrary{calc}

\begin{document}

\section{Labelled Transition System for repeated Head Linear Reduction}

\subsection{Notes}
State is a tuple $\langle$ $\lambda$-term with underlined node, context, list of arguments $\rangle$, where
\begin{itemize}
\item $\lambda$-term (a tree; by considering $\lambda$-term as a tree it becomes possible to cross arguments out of tree ($\dots without\ term$)) with underlined node is a usual lambda term with one underlined position;
\item context $\Gamma$ is an unordered list of pair ($variable : term$);
\item list of arguments $\Delta$ is an ordere list of $\lambda$-terms. (one can also think about $\Delta$ as unordered list of pointers to the corresponding subtree)
\end{itemize}


\subsection{Rules}

$$
\Rule{\State{\InContext{A}{e_1\underline{@}e_2};\;\Gamma;\;\Delta}}
     {\State{\InContext{A}{\underline{e_1}@e_2};\;\Gamma;\;e_2\bullet\Delta}}
\RuleNo{App}
$$

$$
\Rule{\State{\InContext{A}{\underline{\lambda x}.e};\;\Gamma;\;B\bullet\Delta}}
     {\State{\InContext{A_{\xcancel{B}}}{\xcancel{\lambda x}.\underline{e}};\;x:B,\,\Gamma;\;\Delta}}
\RuleNo{Lam-Elim}
$$

$$
\Rule{\State{\InContext{A}{\underline{\lambda x}.e};\;\Gamma;\;\$\bullet\Delta}}
     {\State{\InContext{A}{\lambda x.\underline{e}};\;\Gamma;\;\Delta}}
\RuleNo{Lam-Non-Elim}
$$

$$
\Rule{\State{\InContext{A}{\underline{x}};\;x:B,\,\Gamma;\;\Delta}}
     {\State{\InContext{A}{\underline{B}};\;x:B,\,\Gamma;\;\Delta}}
\RuleNo{BVar}
$$

$$
\Rule{\State{\InContext{A}{\InContext{M}{\underline{x}}@B};\;\Gamma;\;B\bullet\Delta}}
     {\State{\InContext{A}{M@\underline{B}};\;\Gamma;\;\$\bullet\Delta}},\;x\notin dom\,\Gamma
\RuleNo{FVar-Pause-0}
$$

$$
\Rule{\State{\InContext{A}{\InContext{M}{\underline{x}}@B};\;\Gamma;\;\$\bullet B\bullet\Delta}}
     {\State{\InContext{A}{M@\underline{B}};\;\Gamma;\;\$\bullet\Delta}},\;x\notin dom\,\Gamma
\RuleNo{FVar-Pause-1}
$$

$$
\Rule{\State{\InContext{A}{\underline{x}};\;\Gamma;\;\$\bullet\$\bullet\Delta}}
     {\State{\InContext{A}{\underline{x}};\;\Gamma;\;\$\bullet\Delta}},\;x\notin dom\,\Gamma
\RuleNo{FVar-Pause-2}
$$

\end{document}
