\documentclass[a4paper, 10pt]{article}
\usepackage[margin=0.5in]{geometry}
% \usepackage[a4paper, total={6in, 8in}]{geometry}
\usepackage[utf8]{inputenc}%кодировка
\usepackage[russian]{babel}%используем русский и английский языки с переносами
% \graphicspath{{/}}%путь к рисункам
\usepackage{cite}
\usepackage{multirow}
\usepackage{float}
\usepackage{tikz}
\usetikzlibrary{arrows}
\usepackage{ amssymb }
\usepackage{ amsmath}
\usepackage{listings}
\usepackage{xcolor}
\graphicspath{{pictures/}}
\definecolor{dkgreen}{rgb}{0,0.6,0}
\definecolor{dred}{rgb}{0.545,0,0}
\definecolor{dblue}{rgb}{0,0,0.545}
\definecolor{lgrey}{rgb}{0.9,0.9,0.9}
\definecolor{gray}{rgb}{0.4,0.4,0.4}
\definecolor{darkblue}{rgb}{0.0,0.0,0.6}

\usepackage{pythontex}
\usepackage{minted}

% for printing trees
\usepackage{verbatim}

\usepackage[labelformat=empty]{caption}
\usepackage{fontspec}
\usepackage{polyglossia}
\setdefaultlanguage{english}
\defaultfontfeatures{Ligatures=TeX}
\setmainfont{CMU Serif}
\setsansfont{CMU Sans Serif}
\setmonofont{CMU Typewriter Text}

\usepackage{stmaryrd}
\usepackage{amsfonts}
\newcommand\abs[1]{\left|#1\right|}

\usepackage{tikz}
\newcommand{\tikzmark}[3][]{\tikz[remember picture,baseline] \node [inner xsep=0pt,anchor=base,#1](#2) {#3};}
\usepackage{lscape}
\usepackage{pdflscape}

\lstset{
  frame=none,
  xleftmargin=2pt,
  stepnumber=1,
  numbers=left,
  numbersep=5pt,
  numberstyle=\ttfamily\tiny\color[gray]{0.3},
  belowcaptionskip=\bigskipamount,
  captionpos=b,
  escapeinside={*'}{'*},
  language=haskell,
  tabsize=2,
  emphstyle={\bf},
  commentstyle=\it,
  stringstyle=\mdseries\rmfamily,
  showspaces=false,
  keywordstyle=\bfseries\rmfamily,
  columns=flexible,
  basicstyle=\small\sffamily,
  showstringspaces=false,
  morecomment=[l]\%,
}

\usetikzlibrary{calc}

\begin{document}

\section{Labelled Transition System for repeated Head Linear Reduction}

\subsection{Notes}
State is a tuple $\langle$ $\lambda$-term with underlined node, context, list of arguments $\rangle$, where
\begin{itemize}
\item $\lambda$-term (a tree; by considering $\lambda$-term as a tree it becomes possible to cross arguments out of tree ($\dots without\ term$)) with underlined node is a usual lambda term with one underlined position;
\item context $\Gamma$ is an unordered list of pair ($variable : term$);
\item list of arguments $\Delta$ is an ordere list of $\lambda$-terms. (one can also think about $\Delta$ as unordered list of pointers to the corresponding subtree)
\end{itemize}


\subsection{Rules}

\begin{enumerate}
\item (App)

$\dots_1 (e_1 \underline{@} e_2) \dots_2;\ \Gamma;\ \Delta$
$\longrightarrow$
$\dots_1 (\underline{e_1} @ e_2) \dots_2;\ \Gamma;\ e_2:\Delta$

\item (Lam--elim)

$\dots_1 (\underline{\lambda x} . e_1) \dots_2;\ x:B,\ \Gamma;\ B,\ \Delta$
$\longrightarrow$
$\dots_1 (\underline{e_1}) (\dots_2\ without\ B);\ x:B,\ \Gamma;\ \Delta$

\item (Lam--non-elim)

$\dots_1 (\underline{\lambda x} . e_1) \dots_2;\ x:B,\ \Gamma;\ \$,\ \Delta$
$\longrightarrow$
$\dots_1 (\lambda x . \underline{e_1}) \dots_2;\ \Gamma;\ \$,\ \Delta$

\item (BVar)

$\dots_1 \underline{x} \dots_2;\ x:B,\ \Gamma;\ \Delta$
$\longrightarrow$
$\dots_1 \underline{B} \dots_2;\ x:B,\ \Gamma;\ \Delta$

\item (FVar--pause-0)

$\dots_1 (\dots_2 \underline{x} )\dots) @ B \dots_3;\ (x:\_) \not\in \Gamma;\ B,\ \Delta$
$\longrightarrow$
$\dots_1 (\dots_2 x )\dots) @ \underline{B} \dots_3;\ \Gamma;\ \$,\ \Delta$
% $\dots \underline{B} \dots;\ \Gamma;\ \$,\ \Delta$

\item (FVar--pause-1)

$\dots_1 (\dots_2 \underline{x} )\dots) @ B \dots_3;\ (x:\_) \not\in \Gamma;\ \$,\ B,\ \Delta$
$\longrightarrow$
$\dots_1 (\dots_2 x )\dots) @ \underline{B} \dots_3;\ \Gamma;\ \$,\ \Delta$

\item (FVar--pause-2)

$\dots_1 \underline{x} \dots_2;\ (x:\_) \not\in \Gamma;\ \$,\ \$,\ \Delta$
$\longrightarrow$
$\dots_1 \underline{x} \dots_2;\ \Gamma;\ \$,\ \Delta$

\item (FVar--stuck-0)

$\dots \underline{x} \dots;\ (x:\_) \not\in \Gamma;\ \emptyset$
$\longrightarrow$
THE END

\item (FVar--stuck-1)

$\dots \underline{x} \dots;\ (x:\_) \not\in \Gamma;\ \$,\ \dots,\ \$,\ \emptyset$
$\longrightarrow$
THE END

\end{enumerate}











\end{document}