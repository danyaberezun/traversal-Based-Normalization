\documentclass[a4paper, 10pt]{article}
\usepackage[margin=0.5in]{geometry}
\usepackage[utf8]{inputenc}%кодировка
\usepackage[russian]{babel}%используем русский и английский языки с переносами
\usepackage{cite}
\usepackage{multirow}
\usepackage{float}
\usepackage{tikz-cd}
\usetikzlibrary{babel}
\usepackage{amssymb }
\usepackage{amsmath}
\usepackage{listings}
\usepackage{cancel}


\usepackage{tikz}
\usetikzlibrary{arrows}
\usetikzlibrary{calc}
\newcommand{\tikzmark}[3][]{\tikz[remember picture,baseline] \node [inner xsep=0pt,anchor=base,#1](#2) {#3};}

%\usepackage{xcolor}
%\graphicspath{{pictures/}}
%\definecolor{dkgreen}{rgb}{0,0.6,0}
%\definecolor{dred}{rgb}{0.545,0,0}
%\definecolor{dblue}{rgb}{0,0,0.545}
%\definecolor{lgrey}{rgb}{0.9,0.9,0.9}
%\definecolor{gray}{rgb}{0.4,0.4,0.4}
%\definecolor{darkblue}{rgb}{0.0,0.0,0.6}

%\usepackage{pythontex}
%\usepackage{minted}

% for printing trees
\usepackage{verbatim}

%\usepackage[labelformat=empty]{caption}
%\usepackage{fontspec}
%\usepackage{polyglossia}
%\setdefaultlanguage{english}
%\defaultfontfeatures{Ligatures=TeX}
%\setmainfont{CMU Serif}
%\setsansfont{CMU Sans Serif}
%\setmonofont{CMU Typewriter Text}

%\usepackage{stmaryrd}
\usepackage{amsfonts}
\newcommand\abs[1]{\left|#1\right|}

\lstset{
  frame=none,
  xleftmargin=2pt,
  stepnumber=1,
  numbers=left,
  numbersep=5pt,
  numberstyle=\ttfamily\tiny\color[gray]{0.3},
  belowcaptionskip=\bigskipamount,
  captionpos=b,
  escapeinside={*'}{'*},
  language=haskell,
  tabsize=2,
  emphstyle={\bf},
  commentstyle=\it,
  stringstyle=\mdseries\rmfamily,
  showspaces=false,
  keywordstyle=\bfseries\rmfamily,
  columns=flexible,
  basicstyle=\small\sffamily,
  showstringspaces=false,
  morecomment=[l]\%,
}

\newcommand{\State}[1]{\left<{#1}\right>}
\newcommand{\InContext}[2]{{#1}\left[{#2}\right]}
\newcommand{\RuleNo}[1]{\eqno[\textsc{#1}]}
\newcommand{\Rule}[2]{{#1}\longrightarrow{#2}}

%\usetikzlibrary{calc}

\begin{document}

\section{Labelled Transition System for Complete Head Linear Reduction}

\subsection{Notes}
State is a tuple $\langle$ $\lambda$-term with underlined node; context; list of pairs (argument, context) $\rangle$, where
\begin{itemize}
\item $\lambda$-term (a tree; by considering $\lambda$-term as a tree it becomes possible to cross arguments out of tree with a parent application node (denotes as $A_{\xcancel{B}}$)) is a usual lambda term with one underlined node;
\item Context $\Gamma$ is an ordered (first-in) set of elements of the following form ($variable \mapsto (term,\ Context)$);
\item $\Delta$ is an ordered list of pairs ($\lambda$-term, Context) (one can also think about $\Delta$ as an ordered list of pointers to the corresponding subtree stored together with a corresponding context);
\item \textit{Initial} configuration is $\langle \lambda-term\ with\ underlined\ root,\ \emptyset,\ \$ \rangle$;
\item \textit{Final} configuration is $\langle \InContext{M}{\underline{x}} \textnormal{(term in normal form)},\ \Gamma,\ \$ \rangle$;
\item Note, that \emph{Complete Head Linear Reduction} (CHLR) can be seen as usual \emph{Head Linear Reduction} (HLR) that is repeatedly applied to the all arguments;
\item From the below LTS for CHLR.
\end{itemize}


\subsection{Labelled transition system for CHLR}

$$
\Rule{\State{\InContext{A}{e_1\underline{@}e_2};\;\Gamma;\;\Delta}}
     {\State{\InContext{A}{\underline{e_1}@e_2};\;\Gamma;\;(e_2,\Gamma)\bullet\Delta}}
\RuleNo{App}
$$

$$
\Rule{\State{\InContext{A}{\underline{\lambda x}.e};\;\Gamma;\;\$\bullet\Delta}}
     {\State{\InContext{A}{\lambda x.\underline{e}};\;\Gamma;\;\$\bullet\Delta}}
\RuleNo{Lam-Non-Elim}
$$

$$
\Rule{\State{\InContext{A}{\underline{\lambda x}.e};\;\Gamma;\;(B,\Gamma')\bullet\Delta}}
     {\State{\InContext{A_{\xcancel{B}}}{\xcancel{\lambda x}.\underline{e}};\;x \mapsto (B,\Gamma'),\,\Gamma;\;\Delta}}
\RuleNo{Lam-Elim}
$$

$$
\Rule{\State{\InContext{A}{\underline{x}};\;x\mapsto (B,\Gamma'),\,\Gamma;\;\Delta}}
     {\State{\InContext{A}{\underline{B}};\;\Gamma';\;\Delta}}
\RuleNo{BVar}
$$

$$
\Rule{\State{\InContext{A}{\InContext{M}{\underline{x}}@B};\;\Gamma;\;(B,\Gamma')\bullet\$\bullet\Delta}}
     {\State{\InContext{A}{M[x]@\underline{B}};\;\Gamma';\;\$\bullet\Delta}},\;x\notin dom(\Gamma)
\RuleNo{FVar-0}
$$

$$
\Rule{\State{\InContext{A}{\InContext{M}{\underline{x}}@B};\;\Gamma;\;(B,\Gamma')\bullet C\bullet\Delta}}
     {\State{\InContext{A}{M[x]@\underline{B}};\;\Gamma';\;\$\bullet C\bullet\Delta}},\;C\ne\$,\;x\notin dom(\Gamma)
\RuleNo{FVar-1}
$$

$$
\Rule{\State{\InContext{A}{\InContext{M}{\underline{x}}@B};\;\Gamma;\;\$\bullet (B,\Gamma')\bullet\Delta}}
     {\State{\InContext{A}{\InContext{M}{x}@\underline{B}};\;\Gamma';\;\$\bullet\Delta}},\;\;x\notin dom(\Gamma)
\RuleNo{FVar-2}
$$

\subsection{Correctness proof}

Note that labelled transition system for HLR can be obtained by removing [FVAR-*] rules from the LTS for CHLR. In other words, first four rules describe HLR while [FVAR-*] rules describes a repeated application of HLR to the arguments.

\paragraph{LTS for HLR}. \\
\textbf{Remark:} it is obvious that in [Lam-Non-Elim] rule for HLR we can safely replace $\$ \bullet \Delta$ by empty list since there is no rule that pop $\$$ sign from $\Delta$.

$$
\Rule{\State{\InContext{A}{e_1\underline{@}e_2};\;\Gamma;\;\Delta}}
     {\State{\InContext{A}{\underline{e_1}@e_2};\;\Gamma;\;(e_2,\Gamma)\bullet\Delta}}
\RuleNo{App}
$$

$$
\Rule{\State{\InContext{A}{\underline{\lambda x}.e};\;\Gamma;\;[]}}
     {\State{\InContext{A}{\lambda x.\underline{e}};\;\Gamma;\;[]}}
\RuleNo{Lam-Non-Elim}
$$

$$
\Rule{\State{\InContext{A}{\underline{\lambda x}.e};\;\Gamma;\;(B,\Gamma')\bullet\Delta}}
     {\State{\InContext{A_{\xcancel{B}}}{\xcancel{\lambda x}.\underline{e}};\;x \mapsto (B,\Gamma'),\,\Gamma;\;\Delta}}
\RuleNo{Lam-Elim}
$$

$$
\Rule{\State{\InContext{A}{\underline{x}};\;x\mapsto (B,\Gamma'),\,\Gamma;\;\Delta}}
     {\State{\InContext{A}{\underline{B}};\;\Gamma';\;\Delta}}
\RuleNo{BVar}
$$

\paragraph{Expansion function.} 
To prove correctness of HLR we introduce an axiliary expansion function $exp$. This function being applied to some state of LTS for HLR results in some $\lambda$-term which we will call \emph{an expansion}.

\begin{align}
  \label{exp1} exp\; \langle M[\underline{A}];\; \Gamma,\ x \mapsto (B, \Gamma');\; \Delta \rangle
  &= exp\; \langle M[\underline{A}[x / B[\Gamma']]];\; \Gamma;\; \Delta \rangle \\
  \label{exp2} exp\; \langle M_B[\underline{A}];\; \emptyset;\; (B,\Gamma')\bullet\Delta \rangle
  &= exp\; \langle M_{B[\Gamma']}[\underline{A}];\; \emptyset;\; \Delta \rangle\\
  \label{exp3} exp\; \langle M;\; \emptyset;\; [] \rangle &= M \\
  \textnormal{where } B[\Gamma'] &= exp\; \langle \underline{B};\; \Gamma';\; [] \rangle
\end{align}

Note that \eqref{exp1} substitudes term $B[\Gamma']$ instead of all occurence of variable $x$ in the subtree of term $M$ that has an underlined node as a root (i.e. subtree $A$). Each recurcive call of \eqref{exp2} substitudes all variables with respect to the corresponding context exactly in one argument term.

Now consider LTS for HLR. Note that:
\begin{itemize}
\item Only $[Lam-Elim]$ rule can change an expansion;
\item The number of transitions without applying $[Lam-Elim]$ rule is limited since the definition of context, input term has a final size, and only $[Lam-Elim]$ rule expands the context; Hereafter we will denote a sequence of transitions without applying $[Lam-Elim]$ rule by $\overset{*}{\rightarrow}$;
\item The expansion function $exp$ cannot change a path from the root of term to the underlined node since:
  \begin{itemize}
  \item recurcive call \eqref{exp1} can only change the subtree of the current term with underlined node as a root; \hfill{\textbf{(I)}}
  \item recurcive call \eqref{exp2} can only change arguments of the current term that are above the underlined node; Moreover, each application of \eqref{exp2} change exactly one argument (that corresponds to the first element of the pair in $\Delta$) leaving all other arguments and the subtree of the current term with underlined node as a root intact. \hfill{\textbf{(II)}}
  \end{itemize}
\end{itemize}


\paragraph{Theorem}. HLR and HR coincide as follows:

\begin{tikzcd}
  \langle \dots \rangle \arrow{rr}{[Lam-Elim]} \arrow{d}{exp} & &
  \langle M_i;\ \Gamma_i;\ \Delta_i \rangle \arrow{d}{exp}
  \arrow{r}{*} & \langle A[\underline{\lambda x}. e];\ \Gamma;\ (B,\Gamma')\bullet\Delta \rangle
  \arrow{dl}{exp}
  \arrow{rr}{[Lam-Elim]} & & \langle A_{\xcancel{B}}[\xcancel{\lambda x}.\underline{e}];\ x \mapsto (B,\Gamma'),\Gamma;\ \Delta \rangle \arrow{d}{exp} &
  \\
  \dots \arrow{rr}{HR} & & M_i' \arrow[dotted]{rrr}{HR} & & & M_{i+1}'
\end{tikzcd}

\paragraph{Proof} by induction on number of $[Lam-Elim]$ steps.

Base: trivial since first element being added to $\Gamma$ is a head redex by definition.

Induction step. We know that after $i^{th}$ application of rule $[Lam-Elim]$ we got some LTS state $\langle M_i;\ \Gamma_i;\ \Delta_i \rangle$ which expansion is some term $M_i'$. As was mentioned above $\overset{*}{\rightarrow}$ has not change the expansion and the number of steps in $\overset{*}{\rightarrow}$ is finite. Thus, we have two possible cases:

first, LTS got stuck and nothing to prove,

and second, we can apply the $[Lam-Elim]$ rule. In the second case we know the form of the state (following the $[Lam-Elim]$ definition), i.e. $\langle A[\underline{\lambda x}. e];\ \Gamma;\ (B,\Gamma')\bullet\Delta \rangle$ (let us denote it $(i)$), and by assumption the result of applying $[Lam-Elim]$ rule is a state $\langle A_{\xcancel{B}}[\xcancel{\lambda x}.\underline{e}];\ x \mapsto (B,\Gamma'),\Gamma;\ \Delta \rangle$ (let us denote it $(ii)$). Now we have to show that applying function $exp$ to that state results in the same term as a step of head reduction from term $M_i'$. (i.e. in the term $M_{i+1}'$)

Let see what will happend if we apply $exp$ function to state $(i)$:
\begin{align}
  &exp\; \langle A[\underline{\lambda x}.\; e];\ \Gamma;\ (B,\Gamma')\bullet\Delta \rangle & \\
  &\overset{exp}{\rightarrow^{*}}
  exp\; \langle A'[\underline{\lambda x}.\; e[\Gamma]];\ \emptyset;\ (B,\Gamma')\bullet\Delta \rangle &\textnormal{by fully applying \eqref{exp1}} \\
  \label{i3} &\overset{exp}{\rightarrow}
  exp\; \langle A'_{B[\Gamma']}[\underline{\lambda x}.\; e[\Gamma]];\ \emptyset;\ \Delta \rangle &\textnormal{by one step of \eqref{exp2}} \\
  &\overset{exp}{\rightarrow}
  \dots &
\end{align}

to state $(ii)$:
\begin{align}
  &exp \langle A_{\xcancel{B}}[\xcancel{\lambda x}.\; \underline{e}];\ x \mapsto (B,\Gamma'),\Gamma;\ \Delta \rangle & \\
  &\overset{exp}{\rightarrow^{*}}
  exp\;\langle A_{\xcancel{B}}[\xcancel{\lambda x}.\; \underline{e}[\Gamma]];\ x \mapsto (B,\Gamma');\ \Delta \rangle &\textnormal{by elliminating $\Gamma$ following \eqref{exp1}} \\
  \label{ii3} &\overset{exp}{\rightarrow}
  exp\; \langle A_{\xcancel{B}}[\xcancel{\lambda x}.\; \underline{e}[\Gamma][x/B[\Gamma']]];\ \emptyset;\ \Delta \rangle &\textnormal{by one step of \eqref{exp1}} \\
  &\overset{exp}{\rightarrow}
  \dots &
\end{align}

It is easy to see that term in state in \eqref{i3}, $A'_{B[\Gamma']}[\underline{\lambda x}.\; e[\Gamma]]$, has the head redex $(\lambda x,\ B)$ (since context is already empty). Thus, if we apply head reduction to this term, it will results exactly in term $A_{\xcancel{B}}[\xcancel{\lambda x}.\; \underline{e}[\Gamma][x/B[\Gamma']]]$ (by head reduction definition) that is equal to the first component of state \eqref{ii3}. Following \textbf{(I)} and \textbf{(II)}, if we will continue to apply $exp$ function then elliminating the same $\Delta$ in both cases will lead no effect on argument $B$ and underlined subtrees. Thus, $(\lambda x,\ B)$ is a head redex in $M_i'$ and the expansion of state $(ii)$ is $M_{i+1}'$. \hfill{\textbf{Qed}} \\ \\ 

\textbf{Consequences}:
\begin{itemize}
\item if head linear reduction terminates then head reduction terminates;
\item if head reduction terminates then head linear reduction terminates;
\item Since expansion is not able to change a path from the root to the underlined node,
  the first component of the final state of LTS for head linear reduction
  contains a term which leftmost path is a part of the head normal form 
  and the expansion of the final state is a node of the Boehm-tree of the input term; \\
  Thus, a repeated application of head linear reduction to all arguments
  (complete head linear reduction) leeds to the normal form.
\end{itemize}



\section{Labelled Transition System for UNP}
This section contains LTS for UNP. This LTS has the same number of rules that the LTS for CHLR.
Moreover, it is easy to see that rules with the same name corresponds to each other. \\ \\

There are two pointer types:
\begin{itemize}
\item Binder pointer (denoted as {\color{green}$\rightarrow$});
  from the bound variables binder pointer points to the dynamic binder in the history
  and for all other nodes it points points to the parent;
  (these kind of pointer is omitted in these notes since they can
  be easly defined by recurcive procedure)
\item Pointer to the last incomplete application; there are two subtypes
  of this pointer:
  \begin{itemize}
  \item Brown pointer ({\color{brown}$\rightarrow$}) can binds either
    lambda node and application (in this case they arrise as \emph{spine redexes})
    or any other node and some application;
  \item Vioilet pointer ({\color{violet}$\rightarrow$}) has the same meaning
    (pointer to the last incamplete application) together with $\$$
    in LTS for CHLR;
  \item Red pointer ({\color{red}$\rightarrow$}) is used when the real colour
    (brown or violet) is not the metter.    
  \end{itemize}
\end{itemize}

Rules to construct a traversal:
\begin{itemize}
\item $(BVar)$
  \[\tikzmark{1}{}\ \ \tikzmark{}{$\dots$}\ \ \tikzmark{2}{$@$}\ \ \tikzmark{}{$\dots$}\ \ \tikzmark{3}{$\lambda x$}\ \ \tikzmark{}{$\dots$}\ \ \tikzmark{4}{$x$}
  \ \ \tikzmark{}{$\longrightarrow$ where $e$ is is a right child of $@$ $\longrightarrow$}
  \ \ \tikzmark{11}{}\ \ \tikzmark{}{$\dots$}\ \ \tikzmark{12}{$@$}\ \ \tikzmark{}{$\dots$}\ \ \tikzmark{13}{$\lambda x$}\ \ \tikzmark{}{$\dots$}\ \ \tikzmark{14}{$x$}\ \ \tikzmark{15}{$=$}\ \ \tikzmark{16}{$e$}\]
  \begin{tikzpicture}[remember picture,overlay]
    {\color{brown}\draw[->] (3.north) to[bend right] (2.north);}
    {\color{green}\draw[->] (4.south) to[bend left] (3.south);}
    {\color{red}\draw[<-]   ($(4.north)+(0,0.5)$)
      node[above,align=left,black] {$i$}  -- (4);}
    {\color{brown}\draw[->] (13.north) to[bend right] (12.north);}
    {\color{green}\draw[->] (14.south) to[bend left] (13.south);}
    {\color{red}\draw[<-]   ($(14.north)+(0,0.5)$)
      node[above,align=left,black] {$i$}  -- (14);}
    {\color{red}\draw[<-]   ($(14.north)+(0.1,0.5)$)
      node[above,align=left,black] {$$}  -- (16);}
  \end{tikzpicture}
  \\ \\ NB: the colour of the added (red) pointer has to be the same as the colour of the first "red" pointer \\ \\
\item $(Lam-Non-Elim)$
  \[\tikzmark{1}{}\ \ \tikzmark{}{$\dots$}\ \ \tikzmark{2}{$@$}\ \ \tikzmark{}{$\dots$}\ \ \tikzmark{3}{$\lambda x$}
  \ \ \tikzmark{}{$\longrightarrow$ where $e$ is a right child of $@$ $\longrightarrow$}
  \ \ \tikzmark{11}{}\ \ \tikzmark{}{$\dots$}\ \ \tikzmark{12}{$@$}\ \ \tikzmark{}{$\dots$}\ \ \tikzmark{13}{$\lambda x$}\ \ \tikzmark{14}{$e$}\]
  \begin{tikzpicture}[remember picture,overlay]
    {\color{violet}\draw[->] (3.north) to[bend right] (2.north);}
    {\color{violet}\draw[->] (13.north) to[bend right] (12.north);}
    {\color{violet}\draw[->] (14.north) to[bend right] (12.north);}
  \end{tikzpicture}
  \\ \\
\item $(Lam-Elim)$
  \[\tikzmark{1}{}\ \ \tikzmark{}{$\dots$}\ \ \tikzmark{2}{$@_1$}\ \ \tikzmark{}{$\dots$}\ \ \tikzmark{3}{$@_2$}
  \ \ \tikzmark{}{$\dots$}\ \ \tikzmark{4}{$\lambda x$}
  \ \ \tikzmark{}{$\longrightarrow$ where $e$ is a child of $\lambda x$ $\longrightarrow$}
  \ \ \tikzmark{11}{}\ \ \tikzmark{}{$\dots$}\ \ \tikzmark{12}{$@_1$}\ \ \tikzmark{}{$\dots$}
  \ \ \tikzmark{13}{$@_2$}\ \ \tikzmark{}{$\dots$}\ \ \tikzmark{14}{$\lambda x$}\ \ \tikzmark{15}{$e$}\]
  \begin{tikzpicture}[remember picture,overlay]
    {\color{red}\draw[->] (3.north) to[bend right] (2.north);}
    {\color{brown}\draw[->] (4.north) to[bend right] (3.north);}
    {\color{red}\draw[->] (13.north) to[bend right] (12.north);}
    {\color{brown}\draw[->] (14.north) to[bend right] (13.north);}
    {\color{red}\draw[->] (15.north) to[bend right] (12.north);}
  \end{tikzpicture}
  \\ \\ NB: the colour of the added (red) pointer has to be the same as the colour of the first "red" pointer \\ \\
\item $(App)$
  \[\tikzmark{1}{}\ \ \tikzmark{}{$\dots$}\ \ \tikzmark{2}{$@$}
  \ \ \tikzmark{}{$\longrightarrow$ where $e$ is a left child of $@$ $\longrightarrow$}
  \ \ \tikzmark{1}{}\ \ \tikzmark{}{$\dots$}\ \ \tikzmark{12}{$@$}\ \ \tikzmark{13}{$e$}\]
  \begin{tikzpicture}[remember picture,overlay]
    {\color{brown}\draw[->] (13.north) to[bend right] (12.north);}
  \end{tikzpicture}
\item $(FVar-0)$
  \[\tikzmark{1}{}\ \ \tikzmark{}{$\dots$}\ \ \tikzmark{2}{$@_1$}\ \ \tikzmark{}{$\dots$}\ \ \tikzmark{3}{$@_2$}
  \ \ \tikzmark{}{$\dots$}\ \ \tikzmark{4}{$y$}
  \ \ \tikzmark{}{$\longrightarrow$ where $e$ is a right child of $@_2$ $\longrightarrow$}
  \ \ \tikzmark{1}{}\ \ \tikzmark{}{$\dots$}\ \ \tikzmark{12}{$@_1$}\ \ \tikzmark{}{$\dots$}\ \ \tikzmark{13}{$@_2$}
  \ \ \tikzmark{}{$\dots$}\ \ \tikzmark{14}{$y$}\ \ \tikzmark{}{$||$}\ \ \tikzmark{15}{$e$}\]
  \begin{tikzpicture}[remember picture,overlay]
    {\color{violet}\draw[->] (3.north) to[bend right] (2.north);}
    {\color{brown}\draw[->] (4.north) to[bend right] (3.north);}
    {\color{violet}\draw[->] (13.north) to[bend right] (12.north);}
    {\color{brown}\draw[->] (14.north) to[bend right] (13.north);}
    {\color{violet}\draw[->] (15.north) to[bend right] (12.north);}
  \end{tikzpicture}
\item $(FVar-1)$
  \[\tikzmark{1}{}\ \ \tikzmark{}{$\dots$}\ \ \tikzmark{2}{$@_1$}\ \ \tikzmark{}{$\dots$}\ \ \tikzmark{3}{$@_2$}
  \ \ \tikzmark{}{$\dots$}\ \ \tikzmark{4}{$y$}
  \ \ \tikzmark{}{$\longrightarrow$ where $e$ is a right child of $@_2$ $\longrightarrow$}
  \ \ \tikzmark{1}{}\ \ \tikzmark{}{$\dots$}\ \ \tikzmark{12}{$@_1$}\ \ \tikzmark{}{$\dots$}\ \ \tikzmark{13}{$@_2$}
  \ \ \tikzmark{}{$\dots$}\ \ \tikzmark{14}{$y$}\ \ \tikzmark{}{$||$}\ \ \tikzmark{15}{$e$}\]
  \begin{tikzpicture}[remember picture,overlay]
    {\color{brown}\draw[->] (3.north) to[bend right] (2.north);}
    {\color{brown}\draw[->] (4.north) to[bend right] (3.north);}
    {\color{brown}\draw[->] (13.north) to[bend right] (12.north);}
    {\color{brown}\draw[->] (14.north) to[bend right] (13.north);}
    {\color{violet}\draw[->] (15.north) to[bend right] (12.north);}
  \end{tikzpicture}
\item $(FVar-2)$
  \[\tikzmark{1}{}\ \ \tikzmark{}{$\dots$}\ \ \tikzmark{2}{$@_1$}\ \ \tikzmark{}{$\dots$}\ \ \tikzmark{3}{$@_2$}
  \ \ \tikzmark{}{$\dots$}\ \ \tikzmark{4}{$y$}
  \ \ \tikzmark{}{$\longrightarrow$ where $e$ is a right child of $@_2$ $\longrightarrow$}
  \ \ \tikzmark{1}{}\ \ \tikzmark{}{$\dots$}\ \ \tikzmark{12}{$@_1$}\ \ \tikzmark{}{$\dots$}\ \ \tikzmark{13}{$@_2$}
  \ \ \tikzmark{}{$\dots$}\ \ \tikzmark{14}{$y$}\ \ \tikzmark{}{$||$}\ \ \tikzmark{15}{$e$}\]
  \begin{tikzpicture}[remember picture,overlay]
    {\color{red}\draw[->] (3.north) to[bend right] (2.north);}
    {\color{violet}\draw[->] (4.north) to[bend right] (3.north);}
    {\color{red}\draw[->] (13.north) to[bend right] (12.north);}
    {\color{violet}\draw[->] (14.north) to[bend right] (13.north);}
    {\color{violet}\draw[->] (15.north) to[bend right] (12.north);}
  \end{tikzpicture}
  \\ \\ NB: the colour of the "red" pointer is not the metter in this case \\ \\
\end{itemize}

Note that $\Delta$ and $\Gamma$ can be easly defined from UNP LTS state as follows:
\begin{itemize}
\item $\Delta$ can be construct by following incomplete application pointer and storing right childs of all reachable applications in reverse order with corresponding contexts (see next item how to define context from traversal);
\item $\Gamma$ can be construct by following binder pointers, and when faces with a lambda--tokens $\lambda x$ which has a brown pointer to some application $@$, store $x \mapsto (e, \Gamma')$ where $e$ is a right child of apllication $@$ and $\Gamma'$ is a context (which is defined recursively from token $@$).
\end{itemize}






%% \subsection{Correctness proof}

%% \subsubsection{Residual term $\InContext{M}{\underline{x}}$ has no redexes}
%% \paragraph{Proof}
%% We will prove by induction (with step size equal to 2) on number of steps that there is no way for redexes to appear in a path from the root to the current underlined node:
%% \begin{itemize}
%% \item \textbf{Base} $\;$ Straightforward:
%%   \begin{enumerate}
%%   \item $n == 0$, empty term has no redexes;
%%   \item $n == 1$, path from root to itself contains no redexes.
%%   \end{enumerate}
%% \item \textbf{Induction Hypothesis} There is no redexes for all paths with a length less than $n$ ($2 \leq n$).
%% \item \textbf{Induction Step ()}\\
%%   Redex is an \textit{application node with $\lambda$-expression as a left child}. We will show that there is no way to construct such subtree without crossing application node out by case analisys on a current (the underlined one) node form:
%%   \begin{itemize}
%%   \item For $[Lam-*],\ [BVar],\ [FVar-*]$ rules this is obvious: they do not add $\lambda$ as a left or down child of some node.
%%   \item The only interesting case is $[App]$ rule. I.e. the rule that was applied on previous step is an $[App]$-rule.
%%     Here we have two possibilities:
%%     \begin{itemize}
%%     \item $e_1 \ne \lambda\ x$; in this case no redexes appear.
%%     \item $e_1 == \lambda\ x$, for some $x$. In this case a redex appears but by $[App]$-rule we have:\\
%%       $\Rule{\State{\InContext{A}{e_1\underline{@}e_2};\;\Gamma;\;\Delta}}
%%        {\State{\InContext{A}{\underline{e_1}@e_2};\;\Gamma;\;e_2\bullet\Delta}}$.\\
%%       Note, that there \textit{exists} a rule that can be applied next --- $[Lam-Elim]$, and no other rule can be applied. This rule will cross application node with its right subterm out.\\
%%     \end{itemize}
%%   \end{itemize}
%%   Thus, no new redexes can appear.
%% \end{itemize}

%% \subsubsection{Only $\langle \InContext{M}{\underline{x}},\ \Gamma,\ \$ \rangle$ can be a final state}
%% \paragraph{Proof}
%% \begin{itemize}
%% \item It is easy to see that the state $\langle \InContext{M}{\underline{x}},\ \Gamma,\ \$ \rangle$ is a final state: there is no rules that can be applied.
%% \item Let's prove that there is no other final states (with respect to the form of the state). Consider another state which either :
%%   \begin{itemize}
%%   \item If current term is not equal to $\InContext{M}{\underline{x}}$ then either $[App]$ or $[Lam-*]$ can be applied.
%%   (together they have no restrictions on $\Gamma$ and $\Delta$)
%%   \item If $\Delta \ne \$$. Note that $\Delta$ can not be empty because there is no rule that can move $\$$ that is on the bottom of it. Thus, $\Delta = A \dots \$$ where either $A == \$$ or $A \ne \$$. In this case one of $[FVar-*]$ rules can be applied.
%%   \end{itemize}
%% \end{itemize}


%% \section{Small-step semantics for Head Linear Reduction}

%% In this section small-step semantics for head linear reduction is presented.
%% Note that this is not complete HLR.
%% I.e. it has to leed to \underline{head normal form}.

%% \paragraph{Axioms}
%% \begin{align*}
%%   \Theta,\ (\lambda x . s) t\ &\overset{e}{\rightarrow} \Theta(\overset{x}{t}),\ s & [Expansion]\\
%%   \Theta,\ x\ &\overset{r}{\rightarrow} \Theta,\ fresh(\Theta(x)),\; if\ \Theta(x) \downarrow & [Read]
%% \end{align*}

%% \paragraph{Rules}
%% $$ \frac{\Theta.\ s\ \rightarrow \ \Theta^{'}.\ s^{'}}
%% {\Theta,\ st \rightarrow\ \Theta^{'},\ s^{'}t}
%% \hfill [App]$$
%% $$ \frac{\Theta.\ s\ \rightarrow \ \Theta^{'}.\ s^{'} \;\;\; x \not\in N(\Theta)}
%% {\Theta,\ \lambda x.s \rightarrow\ \Theta^{'},\ \lambda x. s^{'}}
%% \hfill [Lam]
%% $$
                 
%% \paragraph{Theorem.}








\end{document}

%% Comutting diagrams
%% \begin{tikzcd}
%% T
%% \arrow[drr, bend left, "x"]
%% \arrow[ddr, bend right, "y"]
%% \arrow[dr, dotted, "{(x,y)}" description] & & \\
%% & X \times_Z Y \arrow[r, "p"] \arrow[d, "q"]
%% & X \arrow[d, "f"] \\
%% & Y \arrow[r, "g"]
%% & Z
%% \end{tikzcd}
