% Document Type: LaTeX 2e
\documentclass[12pt,fleqn,landscape]{article}
% \documentclass[handout]{beamer}



\usepackage{upgreek,latexsym, amssymb, amsmath, amsfonts, mathrsfs} 
\usepackage{eucal,upref,yfonts,eufrak,stmaryrd,graphics,color} 
\usepackage[pdftex]{graphicx}
\usepackage{times}
\usepackage[all]{xy} 
\usepackage{xcolor}


\newcounter{chapter}

\setcounter{chapter}{2}

\usepackage{slide_style_new} 

\usepackage{ndj} 

% MACROS USED IN THE SLIDES

\newcommand{\tP}{\tt p}
\newcommand{\Data}{\mathbb{D}}

\newcommand{\where}{?}





\begin{document}\sffamily\bfseries\boldmath



\newcommand{\arity}[2]{\stackrel{\tt#1}{#2}}


\newcommand{\lexp}{$\lambda$-expression}
\newcommand{\lc}{$\lambda$-calculus}
\newcommand{\lopen}{\Lambda^o}
\newcommand{\lclosed}{\Lambda^\bullet}
\newcommand{\lnf}{\Lambda^{nf}}



\newcommand{\mv}[2]{R_{#2}:=R_{#1}}

\newcommand{\out}{\mbox{\bf out}}
\newcommand{\ifc}{\mbox{\bf if}}
\newcommand{\thenc}{\mbox{\bf then}}
\newcommand{\elsec}{\mbox{\bf else}}
\newcommand{\nil}{\mbox{\bf skip}}
\newcommand{\false}{\mbox{\sl false}}
\newcommand{\true}{\mbox{\sl true}}
\newcommand{\while}{\mbox{\bf while}}
\newcommand{\decl}{\mbox{\sl declassify}}
\newcommand{\dow}{\mbox{\bf do}}
\newcommand{\ew}{\mbox{\bf endw}}
\newcommand{\eif}{\mbox{\bf fi}}
\newcommand{\casec}{\mbox{\bf case}}
\newcommand{\ofc}{\mbox{\bf of}}
\newcommand{\letc}{\mbox{\bf let}}
\newcommand{\inc}{\mbox{\bf in}}


\newcommand{\dobf}[1]{dob(#1)}

\newcommand{\pp}[1]{ ^{#1.}}
\newcommand{\outp}{\mbox{\bf output}}
\newcommand{\inpp}{\mbox{\bf input}}

%\newcommand{\skipc}{\mbox{\bf skip}}
\newcommand{\skipc}{\nil}


\color{Black}\LARGE


\newcommand{\ii}{\item}

\newcommand{\red}[1]{{\color{red}#1}}
\newcommand{\green}[1]{{\color{blue!20!black!30!green}#1}}
\newcommand{\blue}[1]{{\color{blue}#1}}
\newcommand{\brown}[1]{{\color{brown}#1}}
\newcommand{\browm}[1]{{\color{browm}#1}}
\newcommand{\violet}[1]{{\color{violet}#1}}

\newcommand{\lam}[1]{{\color{brown}\textit{\boldmath{#1}}}}


%%%%%%%%%%%%%%%%%%%%%%%



%%%%%%%%%%%%%%%%%%%%%%%%%%%%%%%%%%%%%%%%%%%%%%%%%%%%%%%%%%%%%%%%%%
%%% Title

\thispagestyle{empty}
{\ }\vspace{2ex}
\begin{center} \Huge 
   \rule{18cm}{5pt} \\[1ex]
   \rouge{Partial Evaluation and Normalisation by Traversals} \\
   
   \rule{18cm}{5pt}
\end{center}


\begin{center}\LARGE

Work in progress by:
\vair

\bi
\item 
 \bleu{Daniil Berezun}\\\hair
\vertt{Saint Petersburg State University}\\
\vair\vair

\item 
 \bleu{Neil D. Jones}\\\hair
\vertt{ DIKU, University of Copenhagen (prof.\ emeritus)}\\
\ei
\end{center}


\begin{slide}{Introduction}

The much-studied game semantics for PCF can be thought of

\green{as a PCF interpreter}.

\bc
Ong \cite{ong2015} shows that
\ec

\blue{\textit{a $\lambda$-expression \lam{M}}} can be \red{evaluated} (i.e. normalised) by the algorithm that constructs a \green{traversal} of \lam{M}.
\vair\vair

A \green{\textit{traversal}} is a sequence of
\bi
\ii  \green{subexpressions} of \lam{M}.
  This is a finite set, whose elements we will call \red{tokens}
  \hfill \bleu{(think: \noir{$M$} = program, \noir{tokens} = \underline{program points})}
\vair
\ii Any token may have a \green{back pointers}.
\ei

\vair\vair

With this approach to normalisation: there is \red{\em no need for $\beta$-reduction,
environments, ``thunks'' or ``closures''} to do the evaluation
\vair\vair

\end{slide}



\begin{slide}{start point}
\bi
\ii A view of the Oxford normalisation procedure (ONP for short): It is
\vair

\bc\green{an interpreter for {\lexp}s}
\ec

\ii ONP builds a set of traversal \lam{$\mathfrak{Trav}(M)$}
\bi
\ii Let \lam{$tr \;=\; t_0 \bullet t_1 \bullet \dots \bullet t_n \in \mathfrak{Trav}(M)$} \hfill where \lam{$t_i$} is a \blue{token}
\vair
\ii \blue{Syntax-directed inference rules}:

  \hfill based on syntax of the end-token \lam{$t_n$}
\vair
\ii Action: add 0, 1 or more extensions of \lam{$tr$} to \lam{$\mathfrak{Trav}(M)$}. For each,
\vair
    \bi
    \ii Add a new token \lam{$t'$}, yielding \lam{$tr \bullet t'$}
    \ii Add a back pointer from \lam{$t'$}
    \ei
\ei
\vair

\red{Data types}:
\bi
\ii $tr \in \lam{Tr} = Item^{*}$
\ii $\lam{Item} = subexpression(M) \times Tr$
\ei
\ei

\end{slide}


\begin{slide}{SOME ONP CHARACTERISTICS}

\green{Oxford normaliztion procedure}

\bi
\ii Applies to \blue{simply-typed} {\lexp}s
\ii Begins by translating \lam{$M$} into \blue{$\eta$-long} form
\ii Realises the \green{\underline{compete head linear reduction}} of \lam{$M$}
\ii \red{Correctness:} by game semantics and categories, using \lam{$M$}'s types
\ii ONP uses \red{\underline{no $\beta$-reduction}}: all is based on subexpressions of \lam{$M$}
\ii While running, ONP \red{\underline{does not use the types}} of \lam{$M$} at all
\ei

\vair

\vertt{\underline{\textbf{Goals} of this research}}:
\bi
\ii Extend \green{ONP} to \green{UNP}, for the \textit{\red{untyped}} lambda calculus
\ii \green{Partially evaluate} a normaliser with respect to "static" input \lam{$M$}.

  Use this to \red{compile} \lc $\ $ into a \textit{low-level language}.
\ei

\end{slide}



\begin{slide}{PARTIAL EVALUATION, BRIEFLY}

A partial evaluator is a \blue{program specialiser}. Defining property of $spec$:

$$
\forall p\in \mathit{Programs}\ .\ \forall s, d \in \mathit{Data}\ . \ 
\lsem \lsem spec\rsem (p, s)\rsem (d) = \lsem p\rsem (s, d)
$$


\bi
\ii Given program $p$ and \red{static} data $s$, $spec$ builds a \blue{residual program}
  $$p_s\;\stackrel{def}{=}\;\lsem spec \rsem \ p\ s$$

\ii \blue{Staging transformation}:
\vair

  \bi
  \ii $\lsem p \rsem (s, d)\;\;\;\;\;\;\;\;\;\;\;\;\;$ is a \red{1 stage} computation
  \ii $\lsem \lsem spec \rsem (p, s) \rsem (d)\ $ is a \red{2 stage} computation
  \ei

\vair\vair

\ii Program speedup by \green{precomputation}. Applications: \blue{compiling}, and
  \blue{compiler generation} (from an interpreter, and by self-applying $spec$).
\ei

\end{slide}



\begin{slide}{why partially evaluate $\mbox{NP}$}

\be

\ii The $spec$ equation for a normaliser program \fbox{$\mbox{NP}:\;\Lambda \rightarrow Traversals$}
\vair

\bc
\fbox{$
\forall \lam{$M \in \Lambda$}\ . \ 
\lsem\  \lsem spec\rsem (\mbox{NP}, \lam{$M$})\rsem ()  = \lsem \mbox{NP}\rsem (\lam{$M$})
$}
\ec

\vair

\ii $\lambda$-calculus tradition: \lam{$M$} is self-contained.
\vair

So \vertt{why break normalisation into 2 stages}?
\vair

\be
\ii The specialised output $\mbox{NP}_{\lam{$M$}} = \lsem spec \rsem (\mbox{NP}, \lam{$M$})$
  can be in a \blue{much simplier language} than $\lambda$-calculus.

  Our candidate is some \textbf{\brown{low-level language, LLL}}.
\vair

\ii 2 stages will be natural for \blue{\em semantics-directed compiler generation}.

  LLL can be an intermediate language to express semantics:

  \bi
  \ii $\mbox{NP}_1\;=\; \lsem spec \rsem\; \mbox{NP}\; \lam{M}\;\;\;:\; \Lambda \rightarrow LLL$
  \ii $\mbox{NP}_2\;=\; \lsem \_ \rsem\;\;\; :\; LLL \rightarrow Traversals$ \hfill a \red{semantic} function
  \ei
\ee
\ee

\end{slide}



\begin{slide}{how to partially evaluate ONP}

\be
\vair

\ii \blue{Annotate} parts of normalization procedure as either \green{static} or \red{dynamic}. Variables ranging over
  \be
  \ii\label{onpsyntax} \green{tokens} are \green{static}, \hfill (subexpressions of \lam{$M$}; \blue{finitely many});
\vair

  \ii\label{onpbps} \red{back pointers} are \red{dynamic}; \hfill (\blue{unboundedly} many)
  \ii\label{onptraversals} so the \red{traversal} is \red{dynamic} too.
  \ee
\vair

\ii Computations in normalization proicedure \textbf{NP} are either 
  \green{unfolded} or \red{residualised} (runtime code is generated to \red{do them later})
\vair

  \bi
  \ii Perform \green{fully static} computations  \green{at partial evauation time}.
\vair

  \ii Operations to build or test a traversal: generate \red{residual code}.
  \ei
\ee

\end{slide}




\begin{slide}{The residual program \textbf{ONP}$_M = \lsem \lowercase{spec}\rsem \, \mbox{ONP}\ M$}

ONP is not quite structually inductive but it is \blue{semi-compositional}:\\

\hfill\green{Any recursive \textbf{ONP} call has \underline{\underline{a substructure of \lam{$M$}}} as argument.}

\blue{Consequences}:
\bi
\vair

\ii The partial evaluator can do, at specialisation time,

  \hfill \blue{all of the \textbf{ONP} operations that depend only on \lam{$M$}}
\ii ONP$_{\lam{$M$}}$ performs \green{\underline{no operations at all}} on  lambda expressions
\vair

\ii A specialised program ONP$_{\lam{$M$}}$ contains ``residual code'':
  \bi
  \ii operations to extend the traversal
  \ii operations to follow back pointers
  \ei
\vair

\ii Subexpressions of \lam{$M$} will appear, but are only used as \red{tokens}:

  Tokens are \green{indivisible}: used as labels and for equality comparisons
\ei

\end{slide}



\begin{slide}{Status: our work on simply-typed \lc }

\be
\item We have one version of ONP  in  \blue{\sc Haskell} and another in \blue{\sc Scheme}
\item {\sc Haskell} version includes: \green{ typing; conversion to eta-long form;
  the traversal algorithm itself; and construction of the normalised term}.
\item {\sc Scheme} version: nearly ready to apply automatic partial evaluation.

  Plan: use the \blue{\sc unmix} partial evaluator (Sergei Romanenko).

\item The LLL output size is only \red{linearly larger}
  than \lam{$M$}, $|p_{\lam{$M$}}| = O(|\lam{$M$}|)$
\item We have also have a handwritten a \red{\em generating extension} of ONP.
  $$
  \mbox{If\ }p_M = \lsem \mbox{ONP-gen} \rsem^{\sc scheme} (M)
  \mbox{\ then\ }
  \forall M \ .\ \lsem M\rsem^\Lambda =  \lsem p_M \rsem^{LLL} 
  $$

  {\Large \blue{Now: LLL = is a tiny subset of {\tt scheme}, so the output $p_M$ is  a {\sc scheme}  program.}}
\ee
\end{slide}

\begin{slide}{more work for simply-typed \lc }

\vair\vair

Next steps:
\vair

  \bi
  \ii Produce a generating extension, \green{automatically,} by \blue{specialising
    the specialiser to a $\Lambda$-traverser}, using {\sc unmix}
\vair

  \ii Redefine LLL formally as a clean stand-alone subset of {\sc haskell}
\vair

  \ii Use haskell supersompiler
\vair

  \ii Extend existing approach to \green{programs with input \red{dynamic} data}
  \ei
\end{slide}




\begin{slide}{Status:  our work on the \blue{untyped} \lc }

\be
\ii $UNP$ is a normaliser for \brown{$\Lambda^{untyped}$}.
\vair

\ii $UNP$ has been done in {\sc Haskell} and works on a variety of examples.
\vair

\ii Some of traversal items may have \blue{two back pointers}, in comparison: $ONP$ uses only one.
\vair

\ii As $ONP$, $UNP$ is also defined \green{semi-compositionally} by recursion on syntax of $\lambda$-expression \lam{$M$}.
\vair

\ii By specialising $UNP$, an \blue{arbitrary untyped $\lambda$-expression} can be translated to \lam{LLL}.
\vair

\ii Correctness proof: pending.
\ee

\end{slide}



\begin{slide}{Towards separating programs from data in $\Lambda$}

Also we have a one more direction for research:
\be
\ii An idea is to regard a \green{computation of $\lambda$-expression $M$ on input $d$} as a
  \red{ \blue{two-player game} between the {\sc lll}-codes for $M$ and $d$}.
\ii An intresting examples in this case a is usual $\lambda$-calculus definition of function $mult$
  on Church numerals.
\ii Amaizing fact is that \textbf{Loops from out of nowhere}:
  \bi
  \ii \green{Neither {\tt mul} nor the data contain loops};
  \ii But {\tt mul} function is compiled into \blue{an {\sc lll}-program with two nested loops};
  \ii We also expect that we can do the computation \green{entirely without back pointers}.
  \ei
\ii Right now we are trying to express such program-data games in a \red{\em communicating}
  version of \lam{LLL}. 
\ee

\end{slide}

\begin{slide}{Refenerces}

\end{slide}


\end{document}