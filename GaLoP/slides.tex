% Document Type: LaTeX 2e
\documentclass[12pt,fleqn,landscape]{article}
% \documentclass[handout]{beamer}



\usepackage{upgreek,latexsym, amssymb, amsmath, amsfonts, mathrsfs} 
\usepackage{eucal,upref,yfonts,eufrak,stmaryrd,graphics,color} 
\usepackage[pdftex]{graphicx}
\usepackage{times}
\usepackage[all]{xy} 
\usepackage{xcolor}


\newcounter{chapter}

\setcounter{chapter}{2}

\usepackage{slide_style_new} 

\usepackage{ndj} 

% MACROS USED IN THE SLIDES

\newcommand{\tP}{\tt p}
\newcommand{\Data}{\mathbb{D}}

\newcommand{\where}{?}





\begin{document}\sffamily\bfseries\boldmath



\newcommand{\arity}[2]{\stackrel{\tt#1}{#2}}


\newcommand{\lexp}{$\lambda$-expression}
\newcommand{\lc}{$\lambda$-calculus}
\newcommand{\lopen}{\Lambda^o}
\newcommand{\lclosed}{\Lambda^\bullet}
\newcommand{\lnf}{\Lambda^{nf}}



\newcommand{\mv}[2]{R_{#2}:=R_{#1}}

\newcommand{\out}{\mbox{\bf out}}
\newcommand{\ifc}{\mbox{\bf if}}
\newcommand{\thenc}{\mbox{\bf then}}
\newcommand{\elsec}{\mbox{\bf else}}
\newcommand{\nil}{\mbox{\bf skip}}
\newcommand{\false}{\mbox{\sl false}}
\newcommand{\true}{\mbox{\sl true}}
\newcommand{\while}{\mbox{\bf while}}
\newcommand{\decl}{\mbox{\sl declassify}}
\newcommand{\dow}{\mbox{\bf do}}
\newcommand{\ew}{\mbox{\bf endw}}
\newcommand{\eif}{\mbox{\bf fi}}
\newcommand{\casec}{\mbox{\bf case}}
\newcommand{\ofc}{\mbox{\bf of}}
\newcommand{\letc}{\mbox{\bf let}}
\newcommand{\inc}{\mbox{\bf in}}


\newcommand{\dobf}[1]{dob(#1)}

\newcommand{\pp}[1]{ ^{#1.}}
\newcommand{\outp}{\mbox{\bf output}}
\newcommand{\inpp}{\mbox{\bf input}}

%\newcommand{\skipc}{\mbox{\bf skip}}
\newcommand{\skipc}{\nil}


\color{Black}\LARGE


\newcommand{\ii}{\item}


\newcommand{\red}[1]{{\color{red}#1}}
\newcommand{\green}[1]{{\color{blue!20!black!30!green}#1}}
\newcommand{\blue}[1]{{\color{blue}#1}}
\definecolor{brown}{RGB}{139,69,19}
\newcommand{\brown}[1]{{\color{brown}#1}}
\newcommand{\violet}[1]{{\color{violet}#1}}

\newcommand{\lam}[1]{{\brown{\textit{\boldmath{#1}}}}}


%%%%%%%%%%%%%%%%%%%%%%%



%%%%%%%%%%%%%%%%%%%%%%%%%%%%%%%%%%%%%%%%%%%%%%%%%%%%%%%%%%%%%%%%%%
%%% Title

\thispagestyle{empty}
{\ }\vspace{2ex}
\begin{center} \Huge 
   \rule{18cm}{5pt} \\[1ex]
   \rouge{Partial Evaluation and Normalisation by Traversals} \\
   
   \rule{18cm}{5pt}
\end{center}


\begin{center}\LARGE

Work in progress by:
\vair

\bi
\item 
 \bleu{Daniil Berezun}\\\hair
\vertt{Saint Petersburg State University}\\
\vair\vair

\item 
 \bleu{Neil D. Jones}\\\hair
\vertt{ DIKU, University of Copenhagen (prof.\ emeritus)}\\
\ei
\end{center}


\begin{slide}{Introduction}

The much-studied game semantics for PCF can be thought of

\green{as a PCF interpreter}

\bc
Ong \cite{ong2015} shows that
\ec

\blue{\textit{a $\lambda$-expression \lam{M}}} can be \red{evaluated} 
  by the algorithm that 

  constructs a \green{traversals} of \lam{M}
\vair\vair

A \textit{traversal} is a sequence of
\bi
\ii \red{tokens}, subexpressions of \lam{M}
  
  \hfill \bleu{(think: \noir{$M$} = program, \noir{tokens} = \underline{
  program points})}
\vair
\ii Any token may have a \green{back pointer}
\vair\vair

\ii Oxford normalization procedure (\lam{ONP})
\vair\vair

  \bi
  \ii Constructs \lam{$\mathfrak{Trav}(M)$}
  \ei

\vair\vair
\ei

\end{slide}


\begin{slide}{start point}
\bi
\ii An operational view of the Oxford normalisation procedure: 
\vair

  \bc\vertt{interpreter for {\lexp}s}\ec

  \bi
  \ii Let \lam{$tr \;=\; t_0 \bullet t_1 \bullet \dots \bullet t_n \in 
    \mathfrak{Trav}(M)$} \hfill where \lam{$t_i$} is a \blue{token}
  \vair
  \ii \blue{Syntax-directed inference rules}:
    \bc based on syntax of the end-tokens \lam{$t_n$} \ec
  \vair
  \ii Action: add 0, 1 or more extensions of \lam{$tr$} to \lam{$\mathfrak{
    Trav}(M)$}. For each,
  \vair
      \bi
      \ii Add a new token \lam{$t'$}, yielding \lam{$tr \bullet t'$}
      \ii Add a back pointer from \lam{$t'$}
      \ei
  \ei
\vair

\ii Data types:
  \bi
  \ii $tr \in \lam{Tr} = Item^{*}$
  \ii $\lam{Item} = subexpression(M) \times Tr$
  \ei
\ei

\end{slide}


\begin{slide}{SOME \lam{ONP} CHARACTERISTICS}

Oxford normalization procedure

\bi
\ii Applies to \blue{simply-typed} {\lexp}s
\ii Begins by translating \lam{$M$} into \blue{$\eta$-long} form
\ii Realises the \vertt{\underline{compete} head linear reduction} of \lam{$M$}
\ii Correctness: by game semantics and categories, using \lam{$M$}'s types
\ii ONP uses \vertt{\underline{no $\beta$-reduction}}, environments, $\dots$
\ii While running, ONP \red{\underline{does not use the types}} of \lam{$M$} 
  at all
\ei
\vair\vair

\vertt{\underline{\textbf{Goals} of this research}}:
\bi
\vair

\ii Extend \lam{ONP} to \lam{UNP}, for the \textit{untyped} \lc
\vair

\ii Partially evaluate a normaliser
\ei

\end{slide}


\begin{slide}{PARTIAL EVALUATION, BRIEFLY}

Partial Evaluation $=$ \blue{program specialization}. Defining property of
  $spec$:
  $$
  \forall p\in \mathit{Programs}\ .\ \forall s, d \in \mathit{Data}\ . \ 
  \lsem \lsem spec\rsem (p, s)\rsem (d) = \lsem p\rsem (s, d)
  $$

\bi
\ii Given program $p$ and \red{static} data $s$, $spec$ builds a 
  \blue{residual program}
  $$p_s\;\stackrel{def}{=}\;\lsem spec \rsem \ p\ s$$

\ii Program speedup by \green{precomputation}
\vair

\ii \blue{Staging transformation}:
\vair

  \bi
  \ii $\lsem p \rsem (s, d)\;\;\;\;\;\;\;\;\;\;\;\;\;$ is a \red{1 stage} 
    computation
  \ii $\lsem \lsem spec \rsem (p, s) \rsem (d)\ $ is a \red{2 stage} 
    computation
  \ei
\vair

\ii Applications: \blue{compiling}, and \blue{compiler generation}
\vair

\ii An old idea: \vertt{Semantics directed compiler generation}

\ei

\end{slide}


\begin{slide}{why partially evaluate $\mbox{NP}$}

\be

\ii The $spec$ equation for a normaliser program \fbox{$\mbox{NP}:\;\Lambda 
  \rightarrow Traversals$}
\vair
  \bc
  \fbox{$
  \forall \lam{$M \in \Lambda$}\ . \ 
  \lsem\  \lsem spec\rsem (\mbox{NP}, \lam{$M$})\rsem ()  = \lsem \mbox{NP}
  \rsem (\lam{$M$})$}
  \ec
\vair

\ii $\lambda$-calculus tradition: \lam{$M$} is self-contained
\vair

\vertt{Why break normalisation into 2 stages}?
\vair

  \be
  \ii The specialized output $\mbox{NP}_{\lam{$M$}} = \lsem spec \rsem 
    (\mbox{NP}, \lam{$M$})$ can be in a \blue{much simplier language} than \lc
    \vair

    Our candidate is some \textbf{\brown{low-level language, LLL}}
  \vair

  \ii 2 stages will be natural for \blue{\em semantics-directed compiler 
    generation}

    LLL can be an intermediate language to express semantics:

    \bi
    \vair

    \ii $\mbox{NP}_1\;=\; \lsem spec \rsem\; \mbox{NP}\; \lam{M}\;\;\;:\;
      \Lambda \rightarrow LLL$
    \vair
    
    \ii $\mbox{NP}_2\;=\; \lsem \_ \rsem\;\;\; :\; LLL \rightarrow Traversals$ 
      \hfill a \red{semantic} function
    \ei
  \ee
\ee

\end{slide}


\begin{slide}{how to partially evaluate \lam{ONP}}

\be
\vair\vair

\ii \blue{Annotate} parts of normalization procedure as either \green{static} 
  or \red{dynamic}
  \be
  \ii \green{Tokens} are \green{static} \hfill (subexpressions of \lam{$M$}; 
    \blue{finitely many})
\vair

  \ii \red{Back pointers} are \red{dynamic} \hfill (\blue{unboundedly} many)
\vair

  \ii So the \red{traversal} is \red{dynamic} too
  \ee
\vair\vair\vair

\ii Computations in \lam{ONP} are either \green{unfolded} or \red{residualised}
\vair

  \bi
  \ii Perform \green{fully static} computations  \green{at partial evaluation 
    time}
\vair\vair

  \ii Operations to build or test a traversal: generate \red{residual code}
  \ei
\ee

\end{slide}


\begin{slide}{The residual program \lam{ONP}$_{\lam{$M$}} = 
  \lsem \lowercase{spec}\rsem \, \lam{ONP}\ \lam{$M$}$}

\lam{ONP} is \blue{semi-compositional}:\\

\hfill{Any recursive \lam{ONP} call has \underline{\underline{a substructure 
  of \lam{$M$}}} as argument}
\vair\vair

Consequences:
\bi
\vair

\ii The partial evaluator can do (at specialization time)

  \hfill \blue{all of the \lam{ONP} operations that depend only on \lam{$M$}}
\ii \lam{ONP}$_{\lam{$M$}}$ performs \underline{no operations at all} on 
  {\lexp}s
\vair

\ii A specialized program \lam{ONP}$_{\lam{$M$}}$ contains ``residual code'':
  \bi
  \ii operations to extend the traversal
  \ii operations to follow back pointers
  \ei
\vair

\ii Subexpressions of \lam{$M$} will appear, but are only used as tokens
  \vair

  Tokens are \vertt{indivisible}: used as labels and for equality comparisons
\ei

\end{slide}


\begin{slide}{Status: our work on simply-typed \lc }

\be
\ii We have one version of ONP in  \blue{\sc Haskell} and another in 
  \blue{\sc Scheme}
\ii {\sc Scheme} version: nearly ready to apply automatic partial evaluation

  Plan: use the \blue{\sc unmix} partial evaluator (Sergei Romanenko)

\ii The \lam{LLL} program are only \green{linearly larger} than \lam{$M$}, 
  $|p_{\lam{$M$}}| = O(|\lam{$M$}|)$
\ii Handwritten a \blue{\em generating extension} of \lam{ONP}
  $$
  \mbox{If\ }p_{\lam{$M$}} = \lsem \mbox{ONP-gen} \rsem^{\sc Scheme} 
  (\lam{$M$}) \mbox{\ then\ } \forall \lam{$M$} \ .\ \lsem \lam{$M$}
  \rsem^\Lambda =  \lsem p_{\lam{$M$}} \rsem^{\lam{LLL}} 
  $$

\ii Next steps:
\vair

  \bi
  \ii Produce a generating extension, \green{automatically}, using {\sc unmix}
\vair

  \ii Redefine \lam{LLL} formally as a \textit{clean stand-alone} subset of
    {\sc Haskell}
\vair

  \ii Use {\sc Haskell} supercompiler
\vair

  \ii Extend existing approach to programs with \red{dynamic} input
  \ei
\ee
\end{slide}


\begin{slide}{Status: our work on the \blue{untyped} \lc }

\be
\ii \lam{UNP} is a normaliser for \brown{$\Lambda$}
\vair

\ii \lam{UNP} has been done in {\sc Haskell} and works on a variety of 
  examples
\vair

\ii Some traversal items may have \blue{two back pointers},

  \hfill in comparison: \lam{ONP} uses only one
\vair

\ii As \lam{ONP}, \lam{UNP} is also defined \green{semi-compositionally} 

  \hfill by recursion on syntax of \lexp\  \lam{$M$}
\vair

\ii By specializing \lam{UNP}, an \blue{arbitrary} untyped \lexp\  can be 

  translated to \lam{LLL}
\vair

\ii Correctness proof: pending
\ee

\end{slide}


\begin{slide}{Towards separating programs from data in $\Lambda$}

One more research direction:

\bi
\vair\vair

\ii An idea: consider a \blue{computation} of \lexp\ \lam{$M$} on input 
  $d$ as a \blue{two-player game} between the \lam{LLL}~-codes for \lam{$M$} 
  and $d$
\vair\vair

\ii An interesting example in this case a is usual \lc\  definition of 
  function $mult$ on Church numerals

  \bi
  \ii Amaizing fact: \blue{loops come from out of nowhere}
  \vair

  \ii We also expect that we can do the computation (in this case) 

    \green{entirely without back pointers}
  \ei
\vair\vair

\ii \violet{\em Communicating} version of \lam{LLL}.
\ei

\end{slide}

\begin{slide}{References}

\end{slide}


\end{document}